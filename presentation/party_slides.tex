\documentclass{beamer}
\usetheme{Berkeley}
% \usetheme{Boadilla}
%\usetheme{Madrid}
%\usetheme{Montpellier}
%\usetheme{Warsaw}
%\usetheme{Copenhagen}
%\usetheme{Goettingen}
%\usetheme{Hannover}
%\usetheme{PaloAlto}
%\usetheme{AnnArbor}
%\usetheme{Bergen}

%\usepackage{beamerthemesplit}


\usepackage{amscd,amsxtra,amsthm}
%\usepackage[all]{xy}
%\usepackage{etex}
%\usepackage{pictex}
\usepackage{graphicx}
\usepackage{mathtools}

\theoremstyle{conjecture1}
%\newtheorem{conjecture}[theorem]{Conjecture}
\newtheorem{conjecture1}[theorem]{Conjecture 1}
\theoremstyle{conjecture2}
%\newtheorem{conjecture}[theorem]{Conjecture}
\newtheorem{conjecture2}[theorem]{Conjecture 2}

\def\G{\widetilde{G}}
\def\B{\widetilde{B}}
\def\T{\widetilde{T}}
\def\b{\widetilde{b_* }}
\def\M{\overline{M}}
\def\C{\mathbb{C}}
\def\Q{\mathbb{Q}}
\def\Z{\mathbb{Z}}
\def\F{\mathbb{F}}
\def\I{\mathbb{I}}
\def\Q{\mathbb{Q}}
\def\N{\mathbb{N}}
\def\R{\mathbb{R}}
\def\s{\mbox{\bf s}}
\def\pr{\mbox{\bf p}}
\def\i{\mbox{\bf i}}
\def\k{\mbox{\bf k}}
\def\h{\mbox{\bf h}}
\def\e{\epsilon}
\def\vp{\varpi }
\def\O{\mathcal{O}}
\def\v{\upsilon }
\def\p{\wp }
\def\z{\zeta _\upsilon}
\def\d{\cdot}
\def\c{\bullet}
\def\a{\ast}




\title{How to Guard a Museum}
\author[Brayan Mauricio-Gonzalez and Graham Swain \\ \quad \\ Western Carolina University]{Brayan Mauricio-Gonzalez and Graham Swain}
\date{October 6th, 2022 \\ Math 479}

\begin{document}

\frame{\titlepage}

%%%%%%%%%%%%%%%%%%%%%%%%%%%%%%%%%%%%%%%%%%%%%%%%%%%%%%%%%%%%%%%%%%%%%%%%%%%%%%%%%%%%%%%%%%%%%%%%
%%  Definitions
%%%%%%%%%%%%%%%%%%%%%%%%%%%%%%%%%%%%%%%%%%%%%%%%%%%%%%%%%%%%%%%%%%%%%%%%%%%%%%%%%%%%%%%%%%%%%%%%

\section{Definitions and Examples}

\frame{
    \only<1>{
        \begin{definition}
            A polygon is called \textbf{convex} when all of its interior angles are less than $180^\circ$.
        \end{definition}

        \begin{figure}
            \centering
            \includegraphics[scale=.25]{figures/convex-polygon.pdf}
            \includegraphics[scale=.2]{figures/non-convex-polygon.pdf}
            \caption{Convex polygon on the left, non-convex polygon on the right.}
        \end{figure}
    }

    \only<2>{
        \begin{definition}
            A graph is called \textbf{planar} when all of its vertices and edges are contained in a single
            plane.
        \end{definition}
        
        \begin{figure}
            \centering
            \includegraphics[scale=.25]{figures/planar-graph.pdf}
        \end{figure}
    }

    \only<3>{
        \begin{definition}
            A planar graph $G$ is said to be \textbf{triangulated} when adding another edge to $G$ results
            in a nonplanar graph.
        \end{definition}

        \begin{figure}
            \centering
            \includegraphics[scale=.25]{figures/triangulation.pdf}
        \end{figure}
    }

    \only<4>{
        \begin{definition}
            A proper \textbf{vertex coloring} of a graph $G$ is an assignment of colors to the vertices
            of $G$ such that no two adjacent vertices receive the same color.
        \end{definition}

        \begin{figure}
            \centering
            \includegraphics[scale=.25]{figures/coloring.pdf}
            \caption{This is a 3-coloring of G.}
        \end{figure}
    }
}

%%%%%%%%%%%%%%%%%%%%%%%%%%%%%%%%%%%%%%%%%%%%%%%%%%%%%%%%%%%%%%%%%%%%%%%%%%%%%%%%%%%%%%%%%%%%%%%%
%%  Introduction
%%%%%%%%%%%%%%%%%%%%%%%%%%%%%%%%%%%%%%%%%%%%%%%%%%%%%%%%%%%%%%%%%%%%%%%%%%%%%%%%%%%%%%%%%%%%%%%%

\section{Introduction}

\frame{
    The following problem was coined by Victor Klee in 1973. \vspace{3mm}

    Suppose the manager of a museum wants to place guards to watch every point of the building at all times.
    \begin{itemize}
        \item<2-> Guards are stationed at fixed points
        \item<3-> They have the ability to turn around
    \end{itemize}
    \only<4>{How many guards are needed to achieve this?}
}

\begin{frame}[t]
    To answer this question, we will imagine the walls of the museum are a polygon consisting of $n$ sides. \vspace{3mm}
    
    \only<2->{If the polygon is convex, then we only need one guard to oversee the entire museum.}

    \begin{figure}
        \centering
        \only<3>{\includegraphics[scale=.4]{figures/convex-museum.pdf}}
    \end{figure}
\end{frame}

\frame{
    However, the walls of the museum can take on the shape of {\bf any} closed polygon.

    \begin{figure}
        \centering
        \includegraphics[scale=.14]{figures/floor-plan.png}
        \caption{Floor layout of the second floor of the Met. (From NY Times, 2017)}
    \end{figure}
}

\begin{frame}[t]
    \begin{figure}
        \centering
        \only<1>{\includegraphics[scale=.15]{figures/comb-shape.pdf}}
        \only<2>{\includegraphics[scale=.15]{figures/comb-shape-fg-1.pdf}}
        \only<3-5>{\includegraphics[scale=.15]{figures/comb-shape-fg-2.pdf}}
        \only<6->{\includegraphics[scale=.15]{figures/comb-shape-fg-3.pdf}}
    \end{figure}

    \only<1-4>{
        Let's consider a comb-shaped museum with $n = 3m$ walls.
        \begin{itemize}
            \item<2-> Notice that point 1 can only be seen by a guard stationed in the shaded triangle 
            containing the point
            \item<3-> This applies to the other points 2, 3,..., $m$
        \end{itemize}
        \only<4>{We can see that this requires {\bf at least} $m = \frac{n}{3}$ guards, one for each shaded triangle.}
    }

    \only<5->{
        However, $m$ guards are also sufficient.
        \begin{itemize}
            \item<6-> Since the guards can be placed at the bottom of the triangles
            \item<7-> From there they can watch their own triangle as well as down the hall
        \end{itemize}
        \only<8>{Thus, $\lfloor \frac{n}{3} \rfloor$ guards needed for an $n$-walled museum.}
    }
\end{frame}

%%%%%%%%%%%%%%%%%%%%%%%%%%%%%%%%%%%%%%%%%%%%%%%%%%%%%%%%%%%%%%%%%%%%%%%%%%%%%%%%%%%%%%%%%%%%%%%%
%%  Triangulation
%%%%%%%%%%%%%%%%%%%%%%%%%%%%%%%%%%%%%%%%%%%%%%%%%%%%%%%%%%%%%%%%%%%%%%%%%%%%%%%%%%%%%%%%%%%%%%%%

\section{Triangulation}

\begin{frame}
    \vfill
    \centering
    \begin{beamercolorbox}[sep=8pt,center,shadow=true,rounded=true]{title}
        \usebeamerfont{title}Interior Triangulation of Polygons\par%
    \end{beamercolorbox}

    \begin{figure}
        \centering
        \includegraphics[scale=.35]{figures/triangulation-proof-heading.pdf}
    \end{figure}
    \vfill
\end{frame}

\frame{
    \begin{figure}
        \centering
        \includegraphics[scale=.14]{figures/triangulate-pentagon-split.pdf}
    \end{figure}
}

\begin{frame}[t]
    \begin{proof}[Proof]
        \only<1-4>{
            \only<1->{Let $P$ be a non-convex polygon with $n$ sides, we will prove this using induction.}

            \only<2->{Our base case is $n = 3$, which is a triangle.}

            \only<3->{For our inductive hypothesis, assume $P$ can be triangulated if $n < k$, $k$
            being any integer greater than 3.}

            \only<4>{We will show that $P$ can be triangulated when $n = k$.}
        }

        \only<5>{
            To prove this, we will show that there exists a diagonal that can split $P$ 
            into two smaller polygons that can be triangulated.
        }

        \only<6>{A vertex $A$ is convex if its interior angle is less than $180^\circ$. \\
        We know that the sum of the interior angles of a polygon is $(n-2)180$.}

        \only<7>{By the pigeonhole principle, there must be a convex vertex $A$.
        We can imagine that there are $n$ vertices placed into $n-2$ boxes of $180^\circ$.}

        \only<8>{
            Looking at the two neighbors $B$ and $C$ of $A$. \\
            If the segment $BC$ is entirely in $P$, then we have our diagonal. \\
            If the segment is not entirely in $P$, then $ABC$ contains other vertices.
        }

        \only<9>{
            Slide $BC$ towards $A$ until you hit the last vertex $Z$ in $ABC$.
            Now $AZ$ is within $P$, so we have our diagonal.
        }

        \only<10>{
            Now that we have our diagonal we can split $P$ into smaller polygons, both with side $AZ$,
            with less than $k$ sides. 
            By our inductive hypothesis, both of these polygons can be triangulated.
            Thus $P$ can be triangulated.
        }
        \alt<10>{\qedhere}{\phantom\qedhere}
    \end{proof}

    \begin{figure}
        \centering
        \only<4-7>{\includegraphics[scale=.25]{figures/triangulation-proof.pdf}}
        \only<8>{\includegraphics[scale=.25]{figures/triangulation-proof-dotted.pdf}}
        \only<9>{\includegraphics[scale=.25]{figures/triangulation-proof-diagonal.pdf}}
        \only<10>{\includegraphics[scale=.25]{figures/triangulation-proof-split.pdf}}
    \end{figure}
\end{frame}

%%%%%%%%%%%%%%%%%%%%%%%%%%%%%%%%%%%%%%%%%%%%%%%%%%%%%%%%%%%%%%%%%%%%%%%%%%%%%%%%%%%%%%%%%%%%%%%%
%%  Theorem + Proof
%%%%%%%%%%%%%%%%%%%%%%%%%%%%%%%%%%%%%%%%%%%%%%%%%%%%%%%%%%%%%%%%%%%%%%%%%%%%%%%%%%%%%%%%%%%%%%%%

\section{Art Gallery Theorem}

\frame{
    \begin{theorem}[Art Gallery Theorem]
        For any museum with $n$ walls, $\lfloor \frac{n}{3} \rfloor$ guards suffice.
    \end{theorem}
}

\begin{frame}[t]
    \begin{proof}[Proof]\renewcommand{\qedsymbol}{}
        \only<1-2>{
            Let $P$ be a polygon with $n$ walls. \\
            Because of our lemma, we know we can triangulate $P$. \\
            You can think of $P$ as a planar graph with the corners as vertices and the walls and
            diagonals as edges.
        }

        \only<3>{
            We will now show that $P$ is 3-colorable. 

            For $n=3$, the coloring trivial, as it would be a triangle and  every vertex would be a
            different color.
        }

        \only<4>{
            For $n>3$, pick any two vertices $u$ and $v$ connected by a diagonal and split $P$ along $uv$.
        }

        \only<5>{
            This will give us two smaller triangulated graphs that both contain the edge $uv$.

            Continue in this manner until only triangles remain.
        }
    \end{proof}

    \begin{figure}
        \centering
        \only<1>{\includegraphics[scale=.25]{figures/non-convex-polygon.pdf}}
        \only<2-3>{\includegraphics[scale=.25]{figures/art-gallery-triangulated.pdf}}
        \only<4>{\includegraphics[scale=.25]{figures/art-gallery-triangulated-fb-1a.pdf}}
        \only<5>{\includegraphics[scale=.25]{figures/art-gallery-triangulated-fb-1b.pdf}}
    \end{figure}
\end{frame}

\frame{
    \begin{figure}
        \centering
        \only<1>{\includegraphics[scale=.35]{figures/art-gallery-triangulated-fb-2a.pdf}}
        \only<2>{\includegraphics[scale=.35]{figures/art-gallery-triangulated-fb-2b.pdf}}
        \only<3>{\includegraphics[scale=.35]{figures/art-gallery-triangulated-fb-3a.pdf}}
        \only<4>{\includegraphics[scale=.35]{figures/art-gallery-triangulated-fb-3b.pdf}}
        \only<5>{\includegraphics[scale=.35]{figures/art-gallery-triangulated-fb-4.pdf}}
    \end{figure}
}

\frame{
    \begin{proof}\renewcommand{\qedsymbol}{}
        We can color each triangle using three colors. 
        
        Paste these colorings together to form a 3-coloring of $P$.
    \end{proof}
}

\frame{
    \begin{figure}
        \centering
        \only<1>{\includegraphics[scale=.35]{figures/art-gallery-triangulated-fb-5.pdf}}
        \only<2>{\includegraphics[scale=.35]{figures/art-gallery-triangulated-fb-6.pdf}}
        \only<3>{\includegraphics[scale=.35]{figures/art-gallery-triangulated-fb-7.pdf}}
        \only<4>{\includegraphics[scale=.35]{figures/art-gallery-triangulated-fb-8.pdf}}
        \only<5>{\includegraphics[scale=.35]{figures/art-gallery-triangulated-fb-9.pdf}}
        \only<6>{\includegraphics[scale=.35]{figures/art-gallery-triangulated-fb-10.pdf}}
        \only<7>{\includegraphics[scale=.35]{figures/art-gallery-colored.pdf}}
    \end{figure}
}

\begin{frame}[t]
    \begin{proof}[Proof]
        \only<1->{We choose one of the three colors.}
        
        \only<2->{For every vertex of that color, we assign a guard.}
        
        \only<3->{Since every triangle contains that color, we know that each triangle is guarded.} 
        
        \only<4>{So the whole museum can be guarded by $\lfloor \frac{n}{3} \rfloor$ guards.}
        \alt<4>{\qedhere}{\phantom\qedhere}
    \end{proof}

    \begin{figure}
        \centering
        \only<1>{\includegraphics[scale=.25]{figures/art-gallery-colored.pdf}}
        \only<2->{\includegraphics[scale=.23]{figures/art-gallery-guarded.pdf}}
    \end{figure}
\end{frame}

%%%%%%%%%%%%%%%%%%%%%%%%%%%%%%%%%%%%%%%%%%%%%%%%%%%%%%%%%%%%%%%%%%%%%%%%%%%%%%%%%%%%%%%%%%%%%%%%
%%  Conclusion
%%%%%%%%%%%%%%%%%%%%%%%%%%%%%%%%%%%%%%%%%%%%%%%%%%%%%%%%%%%%%%%%%%%%%%%%%%%%%%%%%%%%%%%%%%%%%%%%

\section{Conclusion}

\begin{frame}[t]
    \vfill
    There are several variants to the art gallery theorem. \\
    For example:
    \begin{itemize}
        \item<2-> We may only want to guard the walls
        \item<3> Guards are stationed at edges
    \end{itemize}
    \vfill
\end{frame}

\begin{frame}[t]
    \vfill
    \begin{figure}
        \centering
        \only<4-5>{\includegraphics[scale=.15]{figures/n4-polygon-fb.pdf}}    
    \end{figure}

    \only<1-3>{
        An unsolved variant states: \vspace{3mm}

        \emph{
            \only<2->{
                Suppose each guard may patrol one wall of the museum, so they walk along their wall and 
                see anything that can be seen from any point along that wall. \vspace{1mm}
            }

            \only<3>{
                How many ``wall guards" do we then need to keep control?
            }
        }
    }

    \only<4>{
        Godfried Toussaint constructed the example of a museum which shows that 
        $\lfloor \frac{n}{4} \rfloor$ guards might be necessary. \vspace{3mm}
    }

    \only<5>{
        This polygon has 28 sides, and only requires 7 wall-guards to guard it.

        It is conjectured that, besides some small values of n, that $\lfloor \frac{n}{4} \rfloor$ is 
        also sufficient, but there is still no proof for it.
    }
    \vfill
\end{frame}

%%%%%%%%%%%%%%%%%%%%%%%%%%%%%%%%%%%%%%%%%%%%%%%%%%%%%%%%%%%%%%%%%%%%%%%%%%%%%%%%%%%%%%%%%%%%%%%%
%%  References
%%%%%%%%%%%%%%%%%%%%%%%%%%%%%%%%%%%%%%%%%%%%%%%%%%%%%%%%%%%%%%%%%%%%%%%%%%%%%%%%%%%%%%%%%%%%%%%%

\frame{
    \frametitle{References}
    \begin{enumerate}{
        \scriptsize  
        \item[{[1]}] M. Aigner and G. Ziegler, ``Proofs From THE BOOK," $3^{rd}$ edition, Springer-Verlag, 2004. 203-205
    }\end{enumerate}
}

\end{document}
